\documentclass{beamer}
\usepackage[russian]{babel}
\usetheme{metropolis}

\usepackage{amsthm}
\setbeamertemplate{theorems}[numbered]

\setbeamercolor{block title}{use=structure,fg=white,bg=gray!75!black}
\setbeamercolor{block body}{use=structure,fg=black,bg=gray!20!white}

\usepackage[T2A]{fontenc}
\usepackage[utf8]{inputenc}

\usepackage{hyphenat}
\usepackage{amsmath}
\usepackage{graphicx}

\AtBeginEnvironment{proof}{\renewcommand{\qedsymbol}{}}{}{}

\title{
Микроэкономика-I
}
\author{
Павел Андреянов, PhD
}

\begin{document}

\maketitle

\section{Кобб Дуглас}

\begin{frame}{Кобб Дуглас}

\begin{definition}
Полезностью Кобб-Дугласа называется:
$$U(x, y) = x^{\alpha} y^{1-\alpha}, \quad \alpha \in (0,1)$$  
\end{definition}

Вспомним, что монотонные преобразования полезности не меняют поведение потребителя. Тогда можно применить логарифм и получить:
$$ U(x, y) = \alpha \log x + (1-\alpha) \log y, \quad \alpha \in (0,1).$$ 
Заметим, что эта функция вогнута!!! 
\end{frame}

\begin{frame}{Кобб Дуглас}

Выпишем Лагранжиан:
$$ \mathcal{L} = \alpha \log x + (1-\alpha) \log y - \lambda (px + qy -I).$$ 

Заметим, что я выставляю знак минус так, чтобы у множителя Лагранжа была интерпретация теневой цены выхода за бюджетное ограничение. Это нам пригодится в следующей лекции, а сейчас просто постарайтесь запомнить.
\end{frame}

\begin{frame}{Кобб Дуглас}

Бездумно выпишем три уравнения:

$\mathcal{L}'_x = \alpha/ x - \lambda p = 0$

$\mathcal{L}'_y = (1-\alpha)/y - \lambda q = 0$

$\mathcal{L}'_{\lambda} = I - p x - qy = 0$

Легко видеть, что они эквивалентны

$\alpha - \lambda p x= 0$

$(1-\alpha) - \lambda q y= 0$

$px + qy - I = 0$

\end{frame}

\begin{frame}{Кобб Дуглас}

Обозначим доли бюджета потраченные на $x$ и $y$ как $s_x= px$ и $s_y = qy$ соответственно, и умножим последнее уравнение на $\lambda$. 

Тогда уравнения становятся еще проще:

$\alpha = \lambda s_x$

$(1-\alpha) = \lambda s_y$

$\lambda s_x + \lambda s_y = \lambda I$

Эту систему можно уже решить в уме. 

Получается, что теневая цена равна $\lambda = 1/I$, а доли бюджета потраченные на каждый товар постоянны и равны $\alpha$ и $1-\alpha$.

Это интуитивно?

\end{frame}

\begin{frame}{Кобб Дуглас}

Пусть полезность имеет следующий вид:
$$U(x,y,z) = \alpha \log x + \beta \log y + \gamma \log z$$ 
а цены равны $p, q, r$ соответственно.

Спрос на каждый товар в Кобб-Дугласе описывается следующими уравнениями:
\begin{gather*}
x^{\ast} = \frac{\alpha}{\alpha + \beta + \gamma} \frac{I}{p}, \quad
y^{\ast} = \frac{\alpha}{\alpha + \beta + \gamma} \frac{I}{q}, \quad
z^{\ast} = \frac{\alpha}{\alpha + \beta + \gamma} \frac{I}{r}
\end{gather*}

Такое лучше запомнить наизусть. Также, постарайтесь ответить, являются ли такие товары нормальными, комплементами или субститутами.

\end{frame}

\begin{frame}{Кобб Дуглас}

Нампомним, что косвенная полезность чувствительна к монотонным преобразованиям, поэтому тут важно какая именно спецификация была изначально дана в задаче. 

Для простоты давайте считать, что это спецификация в логарифмах.

Сосчитаем логарифм спроса на первый товар:
$$\log x^{\ast} = \log \alpha - \log (\alpha + \beta + \gamma) + \log I - \log p$$
Аналогично считается логарифм спроса на другие товары. Теперь надо просто подставить их в полезность.

\end{frame}

\begin{frame}{Кобб Дуглас}

Косвенная полезность в Кобб-Дугласе (с точностью до преобразования) имеет вид
$$V(p,q,r,I) = (\alpha + \beta + \gamma) \log I - \alpha \log p - \beta \log q - \gamma \log r + C $$
Константы $C$ можно, как правило, не выписывать так как они исчезнут при первой же попытке продифференцировать.

Эта формула нам будет очень полезна в будущем...
\end{frame}

\section{Леонтьев}

\section{Конец}

\end{document}