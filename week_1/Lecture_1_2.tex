\documentclass{beamer}
\usepackage[russian]{babel}
\usetheme{metropolis}

\usepackage{amsthm}
\setbeamertemplate{theorems}[numbered]

\setbeamercolor{block title}{use=structure,fg=white,bg=gray!75!black}
\setbeamercolor{block body}{use=structure,fg=black,bg=gray!20!white}

\usepackage[T2A]{fontenc}
\usepackage[utf8]{inputenc}

\usepackage{hyphenat}
\usepackage{amsmath}
\usepackage{graphicx}

\AtBeginEnvironment{proof}{\renewcommand{\qedsymbol}{}}{}{}

\title{
Микроэкономика-I
}
\author{
Павел Андреянов, PhD
}

\begin{document}

\maketitle

\section{Выпуклая оптимизация}

\begin{frame}{Полезность}

В модели полезности (классика) у каждого агента в голове зашита функция полезности, которая переводит любой \textbf{портфель} потребительских товаров в вещественное число, с мистической единицей измерения <<утили>>.

\begin{itemize}
\item 3 куба, 1 круг = 8 утилей
\item 12 конусов = 60 утилей
\item 1 конус, 4 круга = 3 утиля
\end{itemize}

Агенты сравнивают утили и принимают экономические решения дабы их максимизировать. Это самая старая модель, поэтому мы будем называть ее \textbf{классической}.

\end{frame}

\end{document}