\documentclass[twoside,leqno]{article} 

\usepackage{cmap}
\usepackage{graphicx} 
\usepackage{wrapfig}
\usepackage{lastpage}
\usepackage{tikz}
\usepackage{pgfplots}
\usepackage{braket}
\usepackage{verbatim}
\usetikzlibrary{arrows}
\usetikzlibrary{calc,positioning,fit,backgrounds}
\usepackage{amsmath} 
\usepackage[T2A]{fontenc}
\usepackage[utf8]{inputenc}	
\usepackage[english,russian]{babel}
\usepackage{geometry} 
\geometry{top=15mm}
\geometry{bottom=15mm}
\geometry{left=15mm}
\geometry{right=15mm}	
\usepackage{amssymb}
\usepackage{icomma} 
\usepackage{multirow}
\usepackage{mathtext} 
\usepackage{mathrsfs}
\usepackage{mathtools}
\usepackage{fancyhdr}
\usepackage{amsmath}
\pagestyle{fancy}
\fancyhf{}
\renewcommand{\headrulewidth}{0,04mm}

 \usepackage{color,hyperref}
 \usepackage{breakurl}
 \definecolor{darkblue}{rgb}{0.0,0.0,0.3}
 \definecolor{medblue}{rgb}{0.0,0.0,0.6}
 \definecolor{darkgreen}{rgb}{0.0,0.3,0.0}
 \definecolor{medgreen}{rgb}{0.0,0.6,0.0}
 \definecolor{darkred}{rgb}{0.3,0.0,0.0}
 \definecolor{medred}{rgb}{0.6,0.0,0.0}

\usepackage{caption}
\newenvironment{Figure}
 {\par\medskip\noindent\minipage{\linewidth}}
 {\endminipage\par\medskip}
\usepackage{hyperref}
\usepackage{mathtext} 
\usepackage{multicol}
\rhead{Исследовательский поток}
\lhead{Микроэкономика - 1}
%\chead{\date{\today}} 
\cfoot{\thepage} 
\makeatletter % сделать "@" "буквой", а не "спецсимволом" - можно использовать "служебные" команды, содержащие @ в названии
\renewcommand{\headrulewidth}{0,6mm} 

\renewcommand{\maketitle}{\begin{center}
		\noindent{\bfseries\scshape\LARGE\@title}\par
		\noindent {\large\itshape\@author}
		\vskip 2ex\end{center}}
\makeatother
	
\author{Яна Коротова}
\title{Домашнее задание №1}

\begin{document}
	\maketitle
\section{Два отдельных графика}

	\begin{multicols}{2}
			\begin{tikzpicture}[
				scale=5,
				axis/.style={thick, ->, >=stealth'},
				il/.style={thin},
				il 1/.style={very thick},
				d/.style={dashed, thin},
				pile/.style={thick, ->, >=stealth', shorten <=2pt, shorten
					>=2pt},
				every node/.style={color=black}
				]
				% axis
				\draw[axis] (-0.05,0)  -- (.9,0) node(xline)[right] {$x$};
				\draw[axis] (0,-0.05) -- (0,.9) node(yline)[above] {$y$};
				% Lines
				\draw[il] (0,.48) node[left]{$200$} coordinate (a1) -- (.72,0) node[below]{$300$} coordinate (a2)
				(.48,0) node[below]{$200$} coordinate (b1) -- (0,.72) node[left]{$300$} coordinate (b2);
				\filldraw (.7,.2) node{КПВ F}
				(.2,.7) node{КПВ H};
				\end{tikzpicture} \\
				\begin{tikzpicture}[
					scale=5,
					axis/.style={thick, ->, >=stealth'},
					il/.style={thin},
					il 1/.style={very thick},
					d/.style={dashed, thin},
					pile/.style={thick, ->, >=stealth', shorten <=2pt, shorten
						>=2pt},
					every node/.style={color=black}
					]
					% axis
					\draw[axis] (-0.05,0)  -- (.9,0) node(xline)[right] {$x$};
					\draw[axis] (0,-0.05) -- (0,.9) node(yline)[above] {$y$};
					% Lines
					\draw[il] (0,.8) node[left]{$500$} -- (.48,.48) -- (.8,0) node[below]{$500$};
					\draw[d] (.48,.48) -- (0,.48) node[left]{$300$}
					(.48,0) node[below]{$300$} -- (.48,.48); 
					\filldraw (.33,.73) node{$P^W=\cfrac{2}{3}$}
					(.77,.32) node{$P^W=\cfrac{3}{2}$};
					\end{tikzpicture} 
		\end{multicols}
\section{Связанные между собой графики}	

\begin{tikzpicture}[
	scale=5,
	axis/.style={thick, ->, >=stealth'},
	il/.style={thin},
	il 1/.style={very thick},
	d/.style={dashed, thin},
	pile/.style={thick, ->, >=stealth', shorten <=2pt, shorten
		>=2pt},
	every node/.style={color=black}
	]
	% axis 1
	\draw[axis] (-0.05,0)  -- (1,0) node(xline)[right] {$q$};
	\draw[axis] (0,-0.05) -- (0,1) node(yline)[align=center, below, left] {$LAC$\\
	$MC$ \\
	$P$};
	% Lines 1
	\draw[il] (0,0) -- (.8,.8) node[above]{$MC$}
	(.08,.9) node[right]{$LAC_1$} .. controls (.32,.25) .. (.9,.7);
	\draw[d] (.385,.385) coordinate (1) -- (.385,0) node[below]{$q_1$}
	(1) -- (0,.385) node[below, left]{$P^b_1$}
	(1) -- (2.385,.385);
	% axis 2
	\draw[axis] (1.95,0)  -- (3,0) node(xline)[right] {$q$};
	\draw[axis] (2,-0.05) -- (2,1) node(yline)[align=center, below, left] {$LAC$\\
	$MC$ \\
	$P$};
	% Lines 2
	\draw[il] (2,0) -- (2.8,.8) node[above]{$MC$}
	(2.07,.8) node[right]{$LAC_2$} .. controls (2.22,.17) .. (2.9,.6);
	\draw[d] (2.3,.3) coordinate (2) -- (2.3,0) node[below]{$q_2$}
	(2) -- (2,.3) node[below left]{$P^b_2$}
	(2,.3) -- (0,.3);
\end{tikzpicture} \\

\section{Пример формул}

\[
\begin{cases}
	q_1=\cfrac{1}{\beta+2\gamma} \cdot \left(\alpha -p_1\cdot \cfrac{\beta+\gamma}{\beta-\gamma}+p_2\cdot \cfrac{\gamma}{\beta-\gamma}+p_3\cdot \cfrac{\gamma}{\beta-\gamma}\right)\\
	q_2=\cfrac{1}{\beta+2\gamma} \cdot \left(\alpha +p_1\cdot \cfrac{\gamma}{\beta-\gamma}-p_2\cdot \cfrac{\beta+\gamma}{\beta-\gamma}+p_3\cdot \cfrac{\gamma}{\beta-\gamma} \right) \\
	q_3=\cfrac{1}{\beta+2\gamma} \cdot \left(\alpha +p_1\cdot \cfrac{\gamma}{\beta-\gamma}+p_2\cdot \cfrac{\gamma}{\beta-\gamma}-p_3\cdot \cfrac{\beta+\gamma}{\beta-\gamma} \right)
\end{cases}	
\]

Формулы также можно писать внутри 

$$ 
F(x, y) = x^2+\log y,
$$

или внутри строки $F(x, y) = x^2+\log y$.

Вот вам еще матрицы:

\[\det \begin{pmatrix}
    1 & -1 & 0  \\
    \gamma & \beta & \gamma \\
    0 & -1 & 1 
\end{pmatrix}=\beta+2\gamma  \]

\section{Пример таблицы}

\begin{center}
    \begin{tabular}{cc|c|c|c|c|}
        \cline{3-6}
        & & \multicolumn{4}{c|}{$p_b$} \\ 
        \cline{3-6} 
        & & $[80;100]$ & $[60;80]$ & $[30;60]$ & $[0;30]$ \\
        \hline 
        \multicolumn{1}{|c||}{\multirow{4}{*}{$p_a$} } &\multicolumn{1}{ c| }{$[80;100]$} & $p_a+p_b-20$ & $p_a+2p_b-30$ & $p_a+3p_b-40$ & $p_a+4p_b-50$ \\ \cline{2-6}
        \multicolumn{1}{|c||}{} &\multicolumn{1}{ c| }{$[60;80]$} & $2p_a+p_b-30$ & $2p_a+2p_b-40$ &$2p_a+3p_b-50$ & $2p_a+4p_b-60$ \\ \cline{2-6}
        \multicolumn{1}{|c||}{} &\multicolumn{1}{ c| }{$[30;60]$} & $3p_a+p_b-40$ & $3p_a+2p_b-50$ & $3p_a+3p_b-60$& $3p_a+4p_b-70$ \\ \cline{2-6}
        \multicolumn{1}{|c||}{} &\multicolumn{1}{ c| }{$[0;30]$} & $4p_a+p_b-50$ & $4p_a+2p_b-60$ & $4p_a+3p_b-70$ & $4p_a+4p_b-80$\\ \hline
    \end{tabular}
\end{center}

\section{Advanced}

Можно также задавать конкретные функции:

\begin{Figure}
	\centering
	\begin{tikzpicture}
		\tikzset{                                         % for optional style
		every pin/.style={circle, scale=1, pin distance=10pt, inner sep=0pt,  font=\footnotesize},
		small dot/.style={fill=black, circle, scale=0.3}}
		\begin{axis}[      
			width=.5\textwidth,
			height=0.45\textwidth,
			domain=0:1,ymin=0,ymax=1.1,samples=90,legend pos=north east,
			no marks,
		legend entries={%$\mu=0.9$, 
		$\mu = 0.6$, $\mu=0.3$, $\mu=0$}
		,title style={at={(0.5,-0.3)},anchor=south,yshift=-0.1},
		xmajorgrids=true,
		ymajorgrids=true,
		grid style=dashed,
		xlabel = $\alpha$,
		ylabel=$c$,]
		%\addplot[green] {\x / (0.1+0.38*\x)};
		\addplot[medred] {0.38*\x / (0.4+0.38*\x)};
		\addplot[medblue] {0.38*\x / (0.7+0.38*\x)};
		\addplot[medgreen] {0.38*\x / (1+0.38*\x)};
		%\node[small dot, pin=right:{$\gamma=\widetilde{\gamma}$}] at (axis cs: 0.8,0.4){};
		\end{axis}
		\end{tikzpicture}
	% \captionof{figure}{$\eta =0.38$} 
\end{Figure}

\end{document}